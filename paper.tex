% PLEASE USE THIS FILE AS A TEMPLATE
% Check file iosart2x.tex for more examples

% add. options: [seceqn,secthm,crcready]
\documentclass[sw]{iosart2x}
\usepackage{todonotes}
\usepackage{cleveref}
%\usepackage{dcolumn}

%%%%%%%%%%% Put your definitions here
\newcommand{\aw}{AnthroWorks3D}


%%%%%%%%%%% End of definitions

\pubyear{2023}
\volume{0}
\firstpage{1}
\lastpage{1}

\begin{document}

\begin{frontmatter}

%\pretitle{}
\title{ANNO: The Anthropological Notation Ontology}
\runtitle{Anthropological Notation Ontology}
%\subtitle{}

% For one author:
%\author{\inits{N.}\fnms{Name1} \snm{Surname1}\ead[label=e1]{first@somewhere.com}}
%\address{Department first, \orgname{University or Company name},
%Abbreviate US states, \cny{Country}\printead[presep={\\}]{e1}}
\todo{Autorenreihenfolge festlegen}
% Höffner, Heuschkel, Schmiedel, Penne, Fritzsch, Ludwig, Mohaupt, Labudde, Uciteli
% TODO: overleaf
% Two or more authors:
\begin{aug}
\author[B]{\inits{M.}\fnms{Marie} \snm{Heuschkel}\ead[label=e3]{x@somewhere.com}}
\author[A]{\inits{K.}\fnms{Konrad} \snm{Höffner}\ead[label=e1]{konrad.hoeffner@uni-leipzig.de}%
\thanks{Corresponding author. \printead{e1}.}}
\author[B]{\inits{F.}\fnms{Fabian} \snm{Schmiedel}\ead[label=e5]{a@somewhere.com}}
\author[B]{\inits{L.}\fnms{Laura} \snm{Penne}\ead[label=e8]{a@somewhere.com}}
\author[B]{\inits{H. T.}\fnms{Hamjo Tim} \snm{Fritzsch}\ead[label=e9]{c@somewhere.com}}
\author[B]{\inits{A.}\fnms{Andy} \snm{Ludwig}\ead[label=e2]{y@somewhere.com}}
\author[B]{\inits{M.}\fnms{Marleen} \snm{Mohaupt}\ead[label=e4]{z@somewhere.com}}
\author[B]{\inits{D.}\fnms{Dirk} \snm{Labudde}\ead[label=e6]{b@somewhere.com}}
\author[A]{\inits{A.}\fnms{Alexandr} \snm{Uciteli}\ead[label=e7]{c@somewhere.com}}
\address[A]{Institute for Medical Informatics, Statistics and Epidemiology (IMISE), \orgname{Leipzig University},
Saxony, \cny{Germany}\printead[presep={\\}]{e1,e7}}
\address[B]{\orgname{Hochschule Mittweida},
Saxony, \cny{Germany}\printead[presep={\\}]{e2,e3,e4,e5,e6}}
\end{aug}

%\begin{review}{editor}
%\reviewer{\fnms{First} \snm{Editor}\address{\orgname{University or Company name}, \cny{Country}}}
%\reviewer{\fnms{Second} \snm{Editor}\address{\orgname{First University or Company name}, \cny{Country}
%    and \orgname{Second University or Company name}, \cny{Country}}}
%\end{review}
%\begin{review}{solicited}
%\reviewer{\fnms{First} \snm{Solicited reviewer}\address{\orgname{University or Company name}, \cny{Country}}}
%\reviewer{\snm{anonymous reviewer}}
%\end{review}
%\begin{review}{open}
%\reviewer{\fnms{First} \snm{Open Reviewer}\address{\orgname{University or Company name}, \cny{Country}}}
%\end{review}

\begin{abstract}
Anthropology relies on osteometric measurements of human bones, but ...
In this paper, first, we describe the anthropology domain.
Then, we discuss design decisions of modelling ANNO.
Next, we show how ANNO is published for the community.
Finally, we describe the integration of the ontology into \aw{}.
\end{abstract}

\begin{keyword}
\kwd{Ontology}
\kwd{Ontology development}
\kwd{Anthropology}
\end{keyword}

\end{frontmatter}

%%%%%%%%%%% The article body starts:
% TODO notes from meeting, einarbeiten:
% 3 Teile Kernmodule mit Use Cases, weitere Komponenten durch Community





% KH: Ich hab hier mal den Text vom Poster als Basis für die Introduction genommen
\section{Introduction}\label{sec:introduction}
Historic and forensic anthropology gives insights into the behaviour, culture and way of life of humans of different epochs in the past.
Osteometry describes quantitative measurement on the human skeletton.
Anthropology relies on osteometric measurements, which has several problems:
Bones can only be investigated in place, which requires transport over far distances.
This causes wear in the bones, and can lead to different measurements.
This can be alleviated with a digital twin, which can be measured at any location.

\section{Background}
There are various nomenclatures for the description of anatomical landmarks.
These allow a uniform classification of anatomical features and structures:

The \emph{Terminologia anatomica}~\citep{ta2} (TA) is a hierarchy of anatomical landmarks of the entire human body.
From its history, the Latin and Greek designations have emerged to allow standardization.
It is advantageous that these languages no longer show progress and are increasingly losing political power. 
In TA, the order of structures follows the natural anatomy.
An anatomical landmark is identified by an individual identification number, the anatomical term, and the English term.
The identification number is critical because certain landmarks are redundant.
For example, there is the \emph{processus zygomaticus} at the \emph{os frontale} and the \emph{os temporale}.
Synonyms, like \emph{zygomatic bone} and \emph{malar bone}, are also given.
Terms included in the \emph{other} category include eponyms, which are concepts named after people, places or things.
For example, \emph{ossa digitorum pedis} (engl. \emph{phalanges of the foot}) are the small bones of the toes and are also denoted as \emph{ossa phalangea pedis} which are named after the Greek word \emph{phalanx} for a dense rectangular infantry formation.
% chapter 2: Ossa (Bones)

The \emph{Foundational Model of Anatomy} (FMA) ontology~\citep{fma} represents the physical organization of human anatomy.
%It allows the knowledge it contains to be represented in a human-understandable and machine-interpretable way.
However, it does not include Greek and Latin labels, which are frequently used by medical professionals.
Due to hypernyms 90\,\% of the terms are re-designated and only 1\,\% contain definitions.
This has a negative impact on the comprehensibility for users.

Furthermore, in both nomenclatures there is an inconsistent use of singular and plural forms, gaps in lateralized landmarks and numbered series of certain bones (e.g. costae).
In anatomy, it is also important to distinguish between the medical and anthropological viewpoints.
For example, relevant anthropological aspects such as the degree of wear of bones are not considered.
Furthermore, the articular surfaces of the individual bones are particularly important in anthropology.
Not all articular surfaces (e.g., on the ossa metatarsalia) are defined in the nomenclatures, resulting in further gaps.
In some cases, contradictions are also found in the definitions of individual landmarks.
For example, the corpus ossis pubis is placed elsewhere in some sources than in others.
Another problem is that the individual services do not synchronize and do not offer enough synonyms. % which services? 
In addition, the anatomical landmarks are not visually represented and are only selectively labeled in known anatomy atlases.
However, since landmarks are three-dimensional structures, it is essential to provide a marker over the entire landmark and delineate it accurately.
The delineations must also be clearly described textually.
However, there are no standardized definitions in anatomy so far and also in the literature the information is sparse.
For example, in some sources only punctual labels are presented on a figure.
A similar situation exists with osteometric landmarks.
Standardized ones are found only on the cranium and mandible.
However, since there are measurement distances on all bones, the respective start and end points must also be defined.
These are stated differently in the literature and have a high number of synonyms.

For these reasons, it is necessary to develop uniform, textual, visual and detailed descriptive definitions for the individual landmarks, as well as to represent them in a new and clear way in an ontology.
The ontology should be easy to understand for the user.

% TODO: add rdfbones

%\url{https://bioportal.bioontology.org/ontologies/FMA}

\section{Developing ANNO}
The development of ANNO and its integration into \aw{} follows the \emph{three ontology method}~\citep{threeontologymethod} for ontology-based software engineering:
The \emph{domain ontology} describes the domain of skeletal anatomy.
The \emph{task ontology}
The General Formal Ontology (GFO)~\citep{gfo} is chosen as \emph{top-level ontology} and integrates the domain and task ontologies.

% todo: do we have an ontology engineering methodology?
% take care to prevent the problems noted by reviewer 1 in https://semantic-web-journal.net/content/modelling-digital-health-data-examode-ontology-histopathology
% TODO: ontology language, expressivity
% TODO: reuse of ontological entities
% TODO: clear validation mechanism
% TODO: top-level ontology?
% @Alex ist das CSV->OWL Script irgendwo publiziert, kann man das zitieren?
\begin{figure}[h]
\includegraphics[width=\textwidth]{img/smog.png}
\caption{Excerpt of the spreadsheet-based input template used by the anthropologists.}\label{fig:smog}
\end{figure}
The domain experts are provided a spreadsheet-based SMOG~\citep{smog} template by the ontologists, see \cref{fig:smog}.
The spreadsheet is transformed to an OWL 2 ontology consisting of a taxonomy, annotations and some simple axioms on the basis of property restrictions.
% ist das richtig übersetzt?
Sources are categorized into core, root, and occurrence sources.\todo{explain}

% Aus dem Projektbericht übersetzt
\subsection{Selection criteria}
Initially, selection criteria were made for the development of the ontology.
All bones were to be integrated.
For the individual anatomical landmarks, it was necessary to take all those that occur in the TA with the exception of anatomical variants.
However, for the cranium, the selection was reduced because the number of all ossa cranii is too high.
Overall, however, it should include those that are of anthropological relevance, i.e., contribute to navigation, localization, and identification on the bone.
Those landmarks included in the definitions of others were also to be defined.
For bones that lie on the median (e.g., mandible or sternum), bilateral landmarks were defined with side reference in each case.
For osteometric landmarks, all those already established in the core literature were to be taken.
Furthermore, those were to be taken that are relevant for meaningful measurement distances and can be annotated in \aw{}.
For measurement distances, those that are established in the bulk of the literature and core literature as well as necessary for discriminant functions and functions for body height and body weight estimation should be selected.
In addition, it also had to be integrable into \aw{}.
For the functions, those with diagnostic value were taken.
These were discriminant functions for sex determination and regression functions for body height and body weight estimation.

\subsection{Process for creating definitions and measurements}
For the definitions and measurement, a minimum amount of English-language and mostly German-language literature was compared in order to develop the definitions from their information.
Notably, Latin or ancient Greek terms were missing from the English-language literature.
The FMA also contains only English terms.
For this reason, these were also searched out with their synonyms.
Overall, the name of the landmark or survey route had to be noted in English, German, and Latin in singular and plural form, its synonyms in the three languages, the FMA and TA ID, and any information on function and delineation.
Where there were contradictions in the literature, the information was highlighted, reconciled, and logically checked.
For the osteometric landmarks, the abbreviations and their synonyms of the three languages were also provided.
Furthermore, for these as well as for the measured sections, the sources were divided into parent source, core source and occurrence source.
The parent source designates the original source of the landmark.
Under the core source, those were listed where the landmarks are listed and sampled by default.
Occurrence source represents the sources where the landmark occurs for measurements.
For redefinitions of landmarks, a meaningful Latin name had to be selected.
Requirements for this were an included position description (e.g. Punctum superioris capitis femoris as superior located point of the Caput femoris) necessary.
For this, the examination of the Latin grammar was relevant.
For the measurement sections, the type of measurement (e.g., distance measurement) and the measurement instrument were taken in addition to the name of the measurement section.
The subsequently created visual definitions were made in the different anatomical views and represented area markings for the anatomical landmarks and point markings for the osteometric ones.

\subsection{Functions}
The functions were divided into discriminant functions for sex determination and regress functions for body height and body weight estimation.
The goal here was to cover functions for at least one European, African, American, and Asian ethnicity or population.
Names were assigned by the number included in each study, the authors, and the year of publication.
In addition, the function, reference population, aspect (e.g., discriminant function), and sample size with division by gender were noted.
In addition, for the discriminant functions, the thresholds of sex assignment, classification accuracy, and misclassification were important; for the regression functions, the error interval was important.

While there are many textual sources of anthropological systematization, they do not agree in all aspects and there is not a single, formally described, standard.
ANNO links to concepts of the FMA when they exist, but structures them in a different way.

\section{Description of ANNO}

\begin{figure}[h]
\includegraphics[width=0.3\textwidth]{img/modules.pdf}
\caption{The six modules of ANNO.}\label{fig:modules}
\end{figure}

\begin{figure}[h]
\includegraphics[width=\textwidth]{img/axisplane.pdf}
\caption{Anatomical axes and planes.}\label{fig:axisplane}
\end{figure}

The ontology consists of the following core modules, see \cref{fig:modules}:
%For the categories, there were the following spreadsheets: bones, anatomical landmarks, osteometric landmarks, measured distances, function (separately for discriminant and regress functions), anatomical position descriptions, sources used, and osteological terminology.
\subsection{Bones}

\subsection{Anatomical Landmarks}
Anatomical landmarks are defined either on a bone or an anatomical compound structure.
\subsection{Osteometric Landmarks}

\subsection{Teeth}
\subsection{Measurements}
For measurements consist of references to sections, the start point, midpoint, end point, and short definitions.

\subsection{Functions}
Functions are categorized into sex determination and regress functions for body height estimation.
RDF is not optimized for mathematical formulas so we model those as literals.

\paragraph{Sex determination}
% auto translated from the project report, use as basis:
The sex of a specific individual within a population may be estimated using a function on skelettal measurements that is specific to this population.
Based on a threshold value, skeletons are classified into male, probably male, indifferent, probably female, and female.

\paragraph{Regress functions}
Regress functions for body height and body weight estimation. The goal here was to cover functions for at least one European, African, American, and Asian ethnic group or population. Names were assigned by the number included in each study, the authors, and the year of publication. In addition, the function, reference population, aspect (e.g., discriminant function), and sample size with division by gender were noted. In addition, for the discriminant functions, the thresholds of sex assignment, classification accuracy, and misclassification were important; for the regression functions, the error interval was important.

\subsection{Height Functions}
\subsection{Landmarks}
All landmarks are annotated with their name in singular and plural form, reference bone, synonyms, textual and visual definition, FMA and TA ID, and sources.

TODO: Modellierung der Anthropologie-Domäne (u.a. Motivation, related work, was ist neu).
TODO: Dann könnte man Design-Entscheidungen beschreiben (wie und warum wurden bestimmte Sachverhalte so in der Ontologie modelliert).
\section{Publication}
ANNO is published on \url{https://ols.imise.uni-leipzig.de} as well as \url{https://annosaxfdm.de/ontology/}.
TODO: Bereitstellung der Ontologie für die Community (OLS etc.). 

\section{Use Case: Integration into \aw{}}
% this is based on project report but has to be rewritten: less about AW3d and more about the usage of ANNO by it
ANNO is used by \aw{}~\citep{anthroworks3d}.
By combining user-friendly techniques of photogrammetry, insights from user experience research and knowledge from game development, a digital twin is created, which can subsequently be examined virtually.
This enables location-independent and parallel work without wear and tear on the bone.
The examination can be performed as often as desired, even if the skeletal individuals or collections are not available at the institute or have already been reburied.
It also draws its advantage when space is limited.
Thus, it requires only three SLR cameras, sufficient exposure and a computer.
The examination proceeds as follows: After photogrammetry is done and the 3D model is created, the objects are imported into the software and calibrated.
For calibration, there is an individual placeholder for each bone, which must first be selected.
Then the calibrated digital twin is placed in a placeholder skeleton.
Annotations can be made in the detailed view of each bone.
Point, line and area markings are available for selection.
Furthermore, it is possible to make measurements.
You can choose between distance, angle and circumference measurements.
After marking or measurement, an identification number is assigned to each one.
There is also a documentation about the time of the annotation as well as the name of the editor.
The marking or measurement can be classified according to certain categories (e.g. anatomical variation).
In addition, annotations are possible in a text field.
The annotations can be edited at any time afterwards, and the date of editing is recorded together with the author.
The annotations are subsequently displayed in different colors, so that overlaps can be kept apart.
A special feature of the software is the automation of measurements.
The program automatically sets pins for osteometric landmarks in a template view.
The person using the program can move these as desired.
The pins displayed have different color shades.
These differ depending on the relevance.
If the osteometric landmark is present in many measurement sections, its relevance increases and the pin appears darker.
After the pins have been roughly set, they can be refined using various views.
In the views, the individual osteometric landmarks are visible from different views, so that a fine adjustment can be made.
This is followed by the automatic measurement.
The individual values for each measurement section can be downloaded as a CSV file.
In addition, it is now possible to perform sex determinations using discriminant functions.
For this purpose, the required measurement sections can be selected.
The results are only visible in one display.
The automation allows a time saving, whereby more bones or skeletons can be examined in a shorter time.
Advantages over the previous approach are:

\begin{itemize}
\item ontology-oriented generation of placeholders/containers for objects to be imported
\item variable properties for different bone elements and bone types 
\item hierarchy for orientation within an examination project according to the ontology tree
\item categorization of measurements and markers according to specifications from the ontology
\item generation of input forms based on the specified properties and bone types
\item automatic generation of measurements
\end{itemize}

%\subsection{}\label{s1.1}

\section{Conclusion and Future Work}
% Basiert auf dem Fazit aus dem Projektbericht
ANNO introduces uniform definitions for the anatomical landmarks of the human skeletal system as well as osteometric landmarks, measurement distances and functions.
basis for anthropological work
The ontology is interlinked with ... and includes sources for all external definitions.
The ontology is used in \aw{} in place of an old, hard-coded format, which saves development time, separates annotation file compatibility from software versions, eases annotation through hierarchical browsing and increases interoperability.
% following two sentences are machine translated, improve and rewrite
It provides a basis for future training of ontologists, knowledge managers, further research activities and specialization opportunities in this field.
Furthermore, there is interest of the results in forensic, historical and prehistoric anthropology, pathology and medicine as well as in the field of computer science and especially medical informatics.
%Moreover, the ontology can be extended.

can be extended with osteological terms of description
Arten von Erhebungen, Öffnungen, ...
anthropologische ebene draufmodellieren zb form
an anthropological landmark can have different anthropological properties


%\begin{figure}[t]
%\includegraphics{}
%\caption{Figure caption.}\label{f1}
%\end{figure}

%\begin{table*}
%\caption{} \label{t1}
%\begin{tabular}{lll}
%\hline
%&&\\
%&&\\
%\hline
%\end{tabular}
%\end{table*}

\begin{ack}
The ANNO project is co-financed with tax funds on the basis of the budget adopted by the Saxon Parliament.
\end{ack}

\nocite{*}
\bibliographystyle{ios1}
\bibliography{anno}
\end{document}
