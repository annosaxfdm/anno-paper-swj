% PLEASE USE THIS FILE AS A TEMPLATE
% Check file iosart2x.tex for more examples

% add. options: [seceqn,secthm,crcready]
\documentclass[sw]{iosart2x}
\usepackage{todonotes}
%\usepackage{dcolumn}

%%%%%%%%%%% Put your definitions here
\newcommand{\aw}{AnthroWorks3D}


%%%%%%%%%%% End of definitions

\pubyear{2023}
\volume{0}
\firstpage{1}
\lastpage{1}

\begin{document}

\begin{frontmatter}

%\pretitle{}
\title{ANNO: The Anthropological Notation Ontology}
\runtitle{Anthropological Notation Ontology}
%\subtitle{}

% For one author:
%\author{\inits{N.}\fnms{Name1} \snm{Surname1}\ead[label=e1]{first@somewhere.com}}
%\address{Department first, \orgname{University or Company name},
%Abbreviate US states, \cny{Country}\printead[presep={\\}]{e1}}
\todo{Autorenreihenfolge festlegen}
% Two or more authors:
\begin{aug}
\author[B]{\inits{M.}\fnms{Marie} \snm{Heuschkel}\ead[label=e3]{x@somewhere.com}}
\author[A]{\inits{K.}\fnms{Konrad} \snm{Höffner}\ead[label=e1]{konrad.hoeffner@uni-leipzig.de}%
\thanks{Corresponding author. \printead{e1}.}}
\author[B]{\inits{A.}\fnms{Andy} \snm{Ludwig}\ead[label=e2]{y@somewhere.com}}
\author[B]{\inits{M.}\fnms{Marleen} \snm{Mohaupt}\ead[label=e4]{z@somewhere.com}}
\author[B]{\inits{F.}\fnms{Fabian} \snm{Schmiedel}\ead[label=e5]{a@somewhere.com}}
\author[B]{\inits{D.}\fnms{Dirk} \snm{Labudde}\ead[label=e6]{b@somewhere.com}}
\author[A]{\inits{A.}\fnms{Alexandr} \snm{Uciteli}\ead[label=e7]{c@somewhere.com}}
\address[A]{Institute for Medical Informatics, Statistics and Epidemiology (IMISE), \orgname{Leipzig University},
Saxony, \cny{Germany}\printead[presep={\\}]{e1,e7}}
\address[B]{\orgname{Hochschule Mittweida},
Saxony, \cny{Germany}\printead[presep={\\}]{e2,e3,e4,e5,e6}}
\end{aug}

%\begin{review}{editor}
%\reviewer{\fnms{First} \snm{Editor}\address{\orgname{University or Company name}, \cny{Country}}}
%\reviewer{\fnms{Second} \snm{Editor}\address{\orgname{First University or Company name}, \cny{Country}
%    and \orgname{Second University or Company name}, \cny{Country}}}
%\end{review}
%\begin{review}{solicited}
%\reviewer{\fnms{First} \snm{Solicited reviewer}\address{\orgname{University or Company name}, \cny{Country}}}
%\reviewer{\snm{anonymous reviewer}}
%\end{review}
%\begin{review}{open}
%\reviewer{\fnms{First} \snm{Open Reviewer}\address{\orgname{University or Company name}, \cny{Country}}}
%\end{review}

\begin{abstract}
Anthropology relies on osteometric measurements of human bones, but ...
In this paper, first, we describe the anthropology domain.
Then, we discuss design decisions of modelling ANNO.
Next, we show how ANNO is published for the community.
Finally, we describe the integration of the ontology into \aw.
\end{abstract}

\begin{keyword}
\kwd{Ontology}
\kwd{Ontology development}
\kwd{Anthropology}
\end{keyword}

\end{frontmatter}

%%%%%%%%%%% The article body starts:

% KH: Ich hab hier mal den Text vom Poster als Basis für die Introduction genommen
\section{Introduction}\label{sec:introduction}
Historic and forensic anthropology gives insights into the behaviour, culture and way of life of humans of different epochs in the past.
Osteometry, describes quantitative measurement on the human skeletton.
Anthropology relies on osteometric measurements which has several problems:
Bones can only be investigated in place, which requires transport over far distances.
This causes wear in the bones, and can lead to different measurements.
This can be alleviated with a digital twin, which can be measured at any location.

\section{Related Work}
The Foundational Model of Anatomy (FMA) ontology \cite{fma} describes concepts of anatomy.
\url{https://bioportal.bioontology.org/ontologies/FMA}

\section{Ontology Development}
%TODO: lohnt sich nicht als separate Sektion, woanders mit unterbringen
% @Alex ist das CSV->OWL Script irgendwo publiziert, kann man das zitieren?
The domain experts are provided a spreadsheet-based input template by the ontologists,
which is transformed to OWL using a custom script.
% ist das richtig übersetzt?
Sources are categorized into core, root, and occurrence sources.



\section{Ontology Modelling}
While there are many textual sources of anthropological systematization, they do not agree in all aspects and there is not a single, formally described, standard.
ANNO links to concepts of the FMA when they exist, but structures them in a different way.
The ontology consists of the following core modules:
%For the categories, there were the following spreadsheets: bones, anatomical landmarks, osteometric landmarks, measured distances, function (separately for discriminant and regress functions), anatomical position descriptions, sources used, and osteological terminology.
\subsection{Bones}

\subsection{Anatomical Landmarks}
Anatomical landmarks are defined either on a bone or an anatomical compound structure.
\subsection{Osteometric Landmarks}

\subsection{Teeth}
\subsection{Measurements}
For measurements consist of references to sections, the start point, midpoint, end point, and short definitions.

\subsection{Functions}
Functions are categorized into sex determination and regress functions for body height estimation.
RDF is not optimized for mathematical formulas so we model those as literals.

\paragraph{Sex determination}
% auto translated from the project report, use as basis:
The sex of a specific individual within a population may be estimated using a function on skelettal measurements that is specific to this population.
Based on a threshold value, skeletons are classified into male, probably male, indifferent, probably female, and female.

\paragraph{Regress functions}
Regress functions for body height and body weight estimation. The goal here was to cover functions for at least one European, African, American, and Asian ethnic group or population. Names were assigned by the number included in each study, the authors, and the year of publication. In addition, the function, reference population, aspect (e.g., discriminant function), and sample size with division by gender were noted. In addition, for the discriminant functions, the thresholds of sex assignment, classification accuracy, and misclassification were important; for the regression functions, the error interval was important.

\subsection{Height Functions}
\subsection{Landmarks}
All landmarks are annotated with their name in singular and plural form, reference bone, synonyms, textual and visual definition, FMA and TA ID, and sources.

TODO: Modellierung der Anthropologie-Domäne (u.a. Motivation, related work, was ist neu).
TODO: Dann könnte man Design-Entscheidungen beschreiben (wie und warum wurden bestimmte Sachverhalte so in der Ontologie modelliert).
\section{Publication}
ANNO is published on \url{https://ols.imise.uni-leipzig.de} as well as \url{https://annosaxfdm.de/ontology/}.
TODO: Bereitstellung der Ontologie für die Community (OLS etc.). 
\section{Integration into \aw}
TODO: Integration der Ontologie in die Anwendung.
%\subsection{}\label{s1.1}

\section{Conclusion and Future Work}
% Basiert auf dem Fazit aus dem Projektbericht
ANNO introduces uniform definitions for the anatomical landmarks of the human skeletal system as well as osteometric landmarks, measurement distances and functions.
The ontology is interlinked with ... and includes sources for all external definitions.
The ontology is used in \aw in place of an old, hard-coded format, which saves development time, separates annotation file compatibility from software versions, eases annotation through hierarchical browsing and increases interoperability.
% following two sentences are machine translated, improve and rewrite
It provides a basis for future training of ontologists, knowledge managers, further research activities and specialization opportunities in this field.
Furthermore, there is interest of the results in forensic, historical and prehistoric anthropology, pathology and medicine as well as in the field of computer science and especially medical informatics.
%Moreover, the ontology can be extended.

%\begin{figure}[t]
%\includegraphics{}
%\caption{Figure caption.}\label{f1}
%\end{figure}

%\begin{table*}
%\caption{} \label{t1}
%\begin{tabular}{lll}
%\hline
%&&\\
%&&\\
%\hline
%\end{tabular}
%\end{table*}

\begin{ack}
The ANNO project is co-financed with tax funds on the basis of the budget adopted by the Saxon Parliament.
\end{ack}

\nocite{*}
\bibliographystyle{ios1}
\bibliography{anno}
\end{document}
