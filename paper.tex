% PLEASE USE THIS FILE AS A TEMPLATE
% Check file iosart2x.tex for more examples

% add. options: [seceqn,secthm,crcready]
\documentclass[sw]{iosart2x}
\usepackage{todonotes}
%\usepackage{dcolumn}

%%%%%%%%%%% Put your definitions here
\newcommand{\aw}{AnthroWorks3D}


%%%%%%%%%%% End of definitions

\pubyear{2023}
\volume{0}
\firstpage{1}
\lastpage{1}

\begin{document}

\begin{frontmatter}

%\pretitle{}
\title{ANNO: The Anthropological Notation Ontology}
\runtitle{Anthropological Notation Ontology}
%\subtitle{}

% For one author:
%\author{\inits{N.}\fnms{Name1} \snm{Surname1}\ead[label=e1]{first@somewhere.com}}
%\address{Department first, \orgname{University or Company name},
%Abbreviate US states, \cny{Country}\printead[presep={\\}]{e1}}
\todo{Autorenreihenfolge festlegen, ist momentan vom Poster genommen}
% Two or more authors:
\begin{aug}
\author[A]{\inits{K.}\fnms{Konrad} \snm{Höffner}\ead[label=e1]{konrad.hoeffner@uni-leipzig.de}%
\thanks{Corresponding author. \printead{e1}.}}
\author[B]{\inits{A.}\fnms{Andy} \snm{Ludwig}\ead[label=e2]{second@somewhere.com}}
\author[B]{\inits{M.}\fnms{Marie} \snm{Heuschkel}\ead[label=e3]{third@somewhere.com}}
\author[B]{\inits{M.}\fnms{Marleen} \snm{Mohaupt}\ead[label=e4]{fourth@somewhere.com}}
\author[B]{\inits{F.}\fnms{Fabian} \snm{Schmiedel}\ead[label=e5]{fifth@somewhere.com}}
\author[B]{\inits{D.}\fnms{Dirk} \snm{Labudde}\ead[label=e6]{second@somewhere.com}}
\author[A]{\inits{A.}\fnms{Alexandr} \snm{Uciteli}\ead[label=e7]{second@somewhere.com}}
\address[A]{Institute for Medical Informatics, Statistics and Epidemiology (IMISE), \orgname{Leipzig University},
Saxony, \cny{Germany}\printead[presep={\\}]{e1,e7}}
\address[B]{\orgname{Hochschule Mittweida},
Saxony, \cny{Germany}\printead[presep={\\}]{e2,e3,e4,e5,e6}}
\end{aug}

%\begin{review}{editor}
%\reviewer{\fnms{First} \snm{Editor}\address{\orgname{University or Company name}, \cny{Country}}}
%\reviewer{\fnms{Second} \snm{Editor}\address{\orgname{First University or Company name}, \cny{Country}
%    and \orgname{Second University or Company name}, \cny{Country}}}
%\end{review}
%\begin{review}{solicited}
%\reviewer{\fnms{First} \snm{Solicited reviewer}\address{\orgname{University or Company name}, \cny{Country}}}
%\reviewer{\snm{anonymous reviewer}}
%\end{review}
%\begin{review}{open}
%\reviewer{\fnms{First} \snm{Open Reviewer}\address{\orgname{University or Company name}, \cny{Country}}}
%\end{review}

\begin{abstract}
Anthropology relies on osteometric measurements of human bones, but ...
In this paper, first, we describe the anthropology domain.
Then, we discuss design decisions of modelling ANNO.
Next, we show how ANNO is published for the community.
Finally, we describe the integration of the ontology into \aw.
\end{abstract}

\begin{keyword}
\kwd{Ontology}
\kwd{Ontology development}
\kwd{Anthropology}
\end{keyword}

\end{frontmatter}

%%%%%%%%%%% The article body starts:

% KH: Ich hab hier mal den Text vom Poster als Basis für die Introduction genommen
\section{Introduction}\label{sec:introduction}
Historic and forensic anthropology gives insights into the behaviour, culture and way of life of humans of different epochs in the past.
Osteometry, describes quantitative measurement on the human skeletton.
Anthropology relies on osteometric measurements which has several problems:
Bones can only be investigated in place, which requires transport over far distances.
This causes wear in the bones, and can lead to different measurements.
This can be alleviated with a digital twin...

\section{Ontology Modelling}
TODO: Modellierung der Anthropologie-Domäne (u.a. Motivation, related work, was ist neu).
TODO: Dann könnte man Design-Entscheidungen beschreiben (wie und warum wurden bestimmte Sachverhalte so in der Ontologie modelliert).
\section{Publication}
TODO: Bereitstellung der Ontologie für die Community (OLS etc.). 
\section{Integration into \aw}
TODO: Integration der Ontologie in die Anwendung.
%\subsection{}\label{s1.1}

%\begin{figure}[t]
%\includegraphics{}
%\caption{Figure caption.}\label{f1}
%\end{figure}

%\begin{table*}
%\caption{} \label{t1}
%\begin{tabular}{lll}
%\hline
%&&\\
%&&\\
%\hline
%\end{tabular}
%\end{table*}

\begin{ack}
The ANNO project is co-financed with tax funds on the basis of the budget adopted by the Saxon Parliament.
\end{ack}

\nocite{*}
\bibliographystyle{ios1}
\bibliography{anno}
\end{document}
